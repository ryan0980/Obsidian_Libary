\section*{DFS and Connectivity of Graphs}


Recall that the \emph{explore} function starting from a given vertex $v$ finds
all vertices in $G$ reachable from $v$.  Below we give two examples of running $explore$. 
%We represent the running process
%using a \emph{search graph}: 
%we draw a new vertex $v_j$ in the search graph when we start calling $explore~(G, v_j)$; 
%we draw a \emph{tree edge}, shown with solid lines below, if we examine an edge $(v_i, v_j)$ and $visited[j] = 0$;
%we draw a \emph{back edge}, shown with dashed lines below, if we examine an edge $(v_i, v_j)$ and $visited[j] = 1$.

\begin{figure}[h!]
\centering{\input{explore-undirected}}
\caption{Running $explore~(G, v_1)$ on an undirected graph.}
\label{fig:explore-undirected}
\end{figure}

\begin{figure}[h!]
\centering{\input{explore-directed}}
\caption{Running $explore~(G, v_1)$ on an directed graph.}
\label{fig:explore-directed}
\end{figure}

We now define ``connected'' and ``connected component'' to formally reveal the connectivity-structure of graphs.
Let $u,v\in V$. We say $u$ and $v$ are \emph{connected} if and only if there exists a path from $u$ to $v$
\emph{and} there exists a path from $v$ to $u$. 
We note that this definition applies to both directed and undirected graph.
In undirected graph, the existence of a path from $u$ to $v$ implies the
existence of a path from $v$ to $u$. However, this is not necessarily true in directed
graphs. For example, in Figure~\ref{fig:explore-directed}, there exists a path from $v_1$ to $v_5$ but there is no path
from $v_5$ to $v_1$~(so they are not connected).

Let $G = (V, E)$. Let $V_1\subset V$. We say $V_1$ is a \emph{connected component} of $G$,
if and only if (1), for \emph{every pair of} $u,v\in V_1$, $u$ and $v$ are connected,
and (2), $V_1$ is \emph{maximal}, i.e., there does not exists vertex $w\in V\setminus V_1$ such
that $V_1\cup \{w\}$ satisfies condition (1).
For example, in Figure~\ref{fig:explore-directed}, $\{v_1, v_3, v_6\}$ is a connected component;
$\{v_2\}$ is a connected component; 
$\{v_1, v_3\}$ is not a connected component~(as it is not maximal, i.e., does not satisfy condition 2). 

The explore algorithm identifies all vertices that can be reached from $v_i$.
Hence, in the case of undirected graphs, these vertices~(including $v_i$) are pairwise reachable, 
i.e., they form a connected component of $G$~(summarized below). In other words,
$explore(G,v_i)$ identifies the connected component of $G$ that includes $v_i$.

\begin{fact}
For undirected graphs, after $explore(G, v_i)$, the vertices that are marked by $visited$, i.e.,
$\{v_j \mid visited[j] = 1\}$ forms a connected component of $G$ that includes $v_i$.
\end{fact}

However, the above fact does not apply to directed graph: Figure~\ref{fig:explore-directed} gives such an example,
where $\{v_1,v_2,v_3,v_5,v_6\}$ does not form a connected component. Note: in directed graphs
$\{v_j \mid visited[j] = 1\}$ are still those vertices that are reachable from $v_i$; it's just that
they may not be a connected component of $G$.

How to identify \emph{all} connected components of an undirected graph?
We can run above explore algorithm multiple times, each of which starts from an un-explored vertex,
until all vertices are explored. To keep track of which vertices are in which connected
component, we will introduce variable \emph{num-cc} to store the index of current connected component.
We redefine the behavior of $visited$ array: $visited[j] = 0$ still represents that $v_j$ has not yet been explored;
$visited[j] = k$, $k\ge 1$, represents that $v_j$ has been explored and $v_j$ is in the $k$-th connected component.

This new algorithm that traverses all vertices and edges of a graph is named as DFS~(depth first search).
We also slightly changed the explore function, which allows to store which connected component each vertex is in.
The pseudo-codes are given below.

\begin{minipage}{0.8\textwidth}
	\aaA {9}{function DFS ($G = (V, E)$)}\xxx
	\aab {num-cc = 0;}\xxx
	\aab {$visited[i] = 0$, for all $1\le i \le |V|$;}\xxx
	\aaB {5}{for $i = 1 \to |V|$}\xxx
	\aaC {3}{if ($visited[i] = 0$)}\xxx
	\aad {num-cc = num-cc + 1;}\xxx
	\aad {explore ($G, v_i$);}\xxx
	\aac {end if;}\xxx
	\aab {end for;}\xxx
	\aaa {end algorithm;}\xxx
\end{minipage}


\begin{minipage}{0.8\textwidth}
	\aaA {5}{function explore ($G = (V, E), v_i \in V$)}\xxx
	\aab {$visited[i] = $ num-cc;}\xxx
	\aaB {2}{for any edge $(v_i, v_j) \in E$}\xxx
	\aac {if ($visited[j] = 0$): explore ($G, v_j$);}\xxx
	\aab {end for;}\xxx
	\aaa {end algorithm;}\xxx
\end{minipage}

Below we gave examples of running DFS on undirected graphs and undirected graphs.

\begin{figure}[h!]
\centering{\input{dfs-undirected}}
\caption{Running $DFS~(G)$ on an undirected graph.}
\label{fig:dfs-undirected}
\end{figure}

\begin{figure}[h!]
\centering{\input{dfs-directed}}
\caption{Running $DFS~(G)$ on a directed graph.}
\label{fig:dfs-directed}
\end{figure}

DFS runs in $\Theta(|E| + |V|)$ time. This is because, each vertex
is explored exactly once, and each edge is examined exactly once~(in the case of directed graphs)
or exactly twice~(in the case of undirected graphs).

\begin{fact}
For undirected graphs, DFS~($G$) identifies all connected components of $G$:
	$\{v_j \mid visited[j] = k \}$ constitutes the $k$-th connected component of $G$.
\end{fact}

Again, the above fact does not apply to directed graphs: $\{v_j \mid visited[j] = k\}$
are not necessarily form a connected component, although you can still run DFS on directed graphs. In
Figure~\ref{fig:dfs-directed}, $\{v_j \mid visited[j] = 1\}$ gives such an
counter-example.
To prepare revealing the connectivity-structure of directed graphs, we first introduce
a spsecial class of directed graphs, given below.

\section*{Directed Acyclic Graph~(DAG)}

\begin{definition}[DAG]
A directed graph $G = (V, E)$ is \emph{acyclic} if and only if $G$ does not contain cycles.
\end{definition}

\begin{definition}[Linearization / Topological Sorting]
Let $G = (V,E)$ be a directed graph. Let $X$ be an ordering of $V$.
If $X$ satisfies: if $(v_i, v_j)\in E$, then $v_i$ is before $v_j$ in $X$,
then we say $X$ is a linearization~(or toplogical sorting) of $G$.
\end{definition}


See some examples below.

\begin{figure}[h!]
\centering{\input{linearization}}
\caption{Examples of linearization.}
\end{figure}

If a directed graph $G$ admits a linearization, then we say $G$ can be \emph{linearized}.
We now show that linearization is an \emph{equivalent} characterization of DAGs.

\begin{claim}
A directed graph $G$ can be linearized if and only if $G$ is a DAG.
\label{claim:dag}
\end{claim}

\emph{Proof.}  Let's first prove that if $G$ can be linearized, then $G$ is a DAG.
This is equivalent to proving its contraposition: if $G$ contains a cycle, then $G$ cannot be linearized.
Suppose that there exists an cycle $v_{i_1} \to v_{i_2} \to \cdots \to v_{i_k} \to v_{i_1}$ in $G$.
Then the linearization $X$ must satisfy that $v_{i_{j}}$ is before $v_{i_{j+1}}$ for all $j = 1, 2, \cdots, k-1$,
and that $v_{i_{k}}$ is before $v_{i_1}$, in $X$. Clearly, this is not possible.

The other side of the statement, i.e., if $G$ is a DAG, then $G$ can always be linearized, can be proved constructively.
We will design an algorithm~(see below), that constructs a linearization for any DAG. \qed


