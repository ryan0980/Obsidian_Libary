\section*{Graph: Definitions and Representations}

A graph is usually denoted as $G = (V, E)$, where $V$ represents vertices, and $E$ represents edges.
There are two types of graphs, directed ones and undirected ones~(see Figures below).

\begin{figure}[h!]
\centering{\input{graphs}}
\caption{Left: a undirected graph $G = (V, E)$, where $V = \{v_1, v_2, v_3, v_4, v_5\}$,
and $E = \{(v_1, v_2), (v_3, v_2), (v_4, v_3), (v_4, v_2), (v_4, v_5)\}$.
Right: a directed graph $G = (V, E)$, where $V = \{v_1, v_2, v_3, v_4, v_5\}$,
and $E = \{(v_1, v_2), (v_3, v_2), (v_4, v_3), (v_4, v_2), (v_4, v_5), (v_5, v_4)\}$.}
\label{fig:graphs}
\end{figure}

Notice that for an edge $(v_i, v_j)$ in an undirected graph, the order of $v_i$ and $v_j$ are interchangable,
i.e., $(v_i, v_j) = (v_j, v_i)$. In directed graph this is not the case.

%An undirected graph $G = (V, E)$ can be transformed into a directed graph $G' = (V, E')$ by replacing an (undirected) edge $(v_i, v_j) \in E$ as two (directed) edges $(v_i, v_j)$ and $(v_j, v_i)$ in $E'$.

Adjacency matrix and adjacency list are two commonly-used data structures to represent a graph.
Adjacency matrix uses a binary matrix $M$ of size $|V|\times |V|$ to store a graph $G = (V, E)$:
$M[i,j] = 1$ if and only if $(v_i, v_j) \in E$. This definition applies to both directed graphs and undirected graphs.

For undirected graphs, clearly the adjacency matrix $M$ is symmetric. 
If we assume that there is no ``self-loop'' edges in the form of $(v_i, v_i)$, 
then the number of ``1''s in $M$ is exactly $2|E|$ for undirected graph.
The number of ``1''s in $M$ is exactly $|E|$ for directed graph. 

\begin{figure}[h!]
\centering{\input{directed-matrix}}
\caption{Adjacency matrix representation~(directed graph).}
\end{figure}

\begin{figure}[h!]
\centering{\input{undirected-matrix}}
\caption{Adjacency matrix representation~(undirected graph).}
\end{figure}

Adjacency list maintains a list/array $A_i$ for each vertex $v_i \in V$, where
$A_i$ stores $\{v_j \in V \mid (v_i, v_j) \in E\}$, i.e., the adjacent edges/vertices of $v_i$.
A pointer is usually maintained for each vertex $v_i$ that points to the array $A_i$.
Clearly, for undirected graph, $\sum_{v_i \in V} |A_i| = 2|E|$, assuming that there is no ``self-loop'' edges.
For directed graph, $\sum_{v_i \in V} |A_i| = |E|$.

\begin{figure}[h!]
\centering{\input{directed-list}}
\caption{Adjacency list representation~(directed graph).}
\end{figure}

\begin{figure}[h!]
\centering{\input{undirected-list}}
\caption{Adjacency list representation~(undirected graph).}
\end{figure}

Which one is better, adjacency matrix or adjacency list? 
Let's consider some measures.
\vspace*{-\topsep}
\begin{itemize}
\item The space complexity. Clearly, the adjacency matrix needs $\Theta(|V|^2)$ space to store.
The adjacency list can be stored in $\Theta(|V| + |E|)$ space, where $\Theta(|V|)$ is used to
store all $|V|$ pointers, and $\Theta(|E|)$ is used to store all arrays $\{A_i\}$ as we've seen
$\sum_{v_i\in V} |A_i| = |E|$. Therefore, for the space complexity, adjacency list is better,
as $|E| = O(|V|^2)$; in particular in \emph{sparse} graphs, $|E|$ will be way smaller than $|V|^2$.

\item Querying if $(v_i,v_j)\in E$. Given $v_i$ and $v_j$, whether $(v_i,v_j)\in E$ can be done
in $\Theta(1)$ time if the graph $G$ is represented with an adjacency matrix, as this can be answered
by a direct access to $M[i,j]$.
If $G$ is represented with an adjacency list, to check if $(v_i,v_j)\in E$, one needs to tranverse
$A_i$ and see if $v_j$ is in $A_i$ or not, and clearly this takes $\Theta(|A_i|)$ time.
Note that, if we assume that the vertices in $A_i$ are sorted, then searching for the appearance of $v_j$
in $A_i$ can be done through \emph{binary search} which takes $\Theta(\log |A_i|)$ time.
In any case, adjacency matrix is better in fastly querying.

\item Listing adjacent vertices of a vertex. Given $v_i$, one needs to tranverse the entire row~(the $i$-th row) of the adjacency matrix
to find all adjacent vertices of $v_i$ which takes $\Theta(|V|)$ time.
In case of adjacency list, one just needs to return the list $A_i$ which takes $\Theta(|A_i|)$ time.
Hence, adjacency list is better in listing adjacent vertices.

\end{itemize}

In practice, adjacency list is usually the first choice, for an obvious reason that, for huge graphs, it is not possible
to store an $|V|\times |V|$ matrix in memory.

\section*{Connectivity of Graphs}

A \emph{path} in a graph is a list of consecutive edges.
In directed graphs, two consecutive edges in a path must be in the form of $(v_a, v_b), (v_b, v_c)$,
i.e., the head of the predecessor connects to the tail of successor. If there exists a path from $u$ to $v$,
then we also say $u$ can reach $v$, or $v$ is reachable from $u$, or $v$ can be reached from $u$.
These definitions applies to both directed and undirected graphs.

%Let $u,v\in V$. We say $u$ and $v$ are \emph{connected} if and only if there exists a path from $u$ to $v$
%and there exists a path from $v$ to $u$. In undirected graph, the existence of a path from $u$ to $v$ implies the
%existence of a path from $v$ to $u$. It's not necessarily the case in directed
%graphs. For example, in Figure~\ref{fig:explore-directed}, there exists a path from $v_1$ to $v_5$ but there is no path
%from $v_5$ to $v_1$~(so they are not connected).
%
%Let $G = (V, E)$. Let $V_1\subset V$. We say $V_1$ is a \emph{connected component} of $G$,
%if and only if (1), for \emph{every pair of} $u,v\in V_1$, $u$ and $v$ are connected,
%and (2), $V_1$ is \emph{maximal}, i.e., there does not exists vertex $w\in V\setminus V_1$ such
%that $V_1\cup \{w\}$ satisfies condition (1).
%For example, in Figure~\ref{fig:explore-directed}, $\{v_1, v_3, v_6\}$ is a connected component;
%$\{v_2\}$ is a connected component; 
%$\{v_1, v_3\}$ is not a connected component~(as it is not maximal). 

%Below, we will design algorithms to reveal the connectivity-structures of graphs, i.e.,
%what their connected components are and how these connected components are connected.

One basic procedure in graphs is to find the set of vertices that are reachable from a given vertex.
We will use an array, called $visited$, of size $|V|$, to store the vertices that are reachable from
the given vertex $v_i$: $visited[j] = 1$ if and only if there exists a path from $v_i$ to $v_j$.
This array will be initialized as 0 for all entries.
The following recursive algorithm, named \emph{explore}, finds all vertices that are reachable from $v_i$
and stores these vertices in $visited$ array properly.

\begin{minipage}{0.8\textwidth}
	\aaA {5}{function explore ($G = (V, E), v_i \in V$)}\xxx
	\aab {$visited[i] = 1$;}\xxx
	\aaB {2}{for any edge $(v_i, v_j) \in E$}\xxx
	\aac {if ($visited[j] = 0$): explore ($G, v_j$);}\xxx
	\aab {end for;}\xxx
	\aaa {end algorithm;}\xxx
\end{minipage}

