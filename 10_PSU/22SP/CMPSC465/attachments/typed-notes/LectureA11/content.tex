\section*{DFS with Timing and Properties}

Recall that the DFS-with-timing algorithm gives an interval $[pre,post]$ for each vertex.
For two vertices $v_i,v_j\in V$, their corresponding intervals can either be
\emph{disjoint}, i.e., the two intervals do not overlap, or \emph{nested}, i.e.,
one interval is within the other. See Figure~\ref{fig:interval}.
But two intervals cannot be \emph{partially overlapping}. Why? This is because
the explore funtion is recursive. There are only two possiblities
that $pre[i] < pre[j]$. The first one is that explore $v_j$ is \emph{within} explore $v_i$;
in this case the recursive behaviour of explore leads to that $post[j] < post[i]$,
as explore $v_j$ must return/terminate first and then explore $v_i$ will return/terminate.
This case corresponds to that the two intervals are nested.
The second one is that explore $v_j$ starts after explore $v_i$ finishes;
this case corresponds to that the two intervals are disjoint.

\begin{figure}[h!]
\centering{\input{interval}}
\caption{Relations between two $[pre,post]$ intervals.}
\label{fig:interval}
\end{figure}

We now show and prove a key claim that relates the post values and the meta-graph.

\begin{claim}
Let $C_i$ and $C_j$ be two connected components of directed graph $G = (V, E)$, i.e., $C_i$ and $C_j$ are two
vertices in its coresponding meta-graph $G_M = (V_M, E_M)$. If we have $(C_i, C_j) \in E_M$ then
we must have that $\max_{u\in C_i} post[u] > \max_{v\in C_j} post[v]$.
\end{claim}

Intuitively, following an edge in the meta-graph, the largest post value decreases.
Before seeing a formal proof, please see an example in Figure~\ref{fig:dfs}:
the largest post values for $C_1$, $C_2$, $C_3$, and $C_4$ are 9, 6, 10, and 14,
and you may verify that following any edge in the meta-graph, the largest post value always decreases.

\begin{figure}[h!]
\centering{\input{dfs}}
\caption{Example of running DFS~(with timing) on a directed graph. The $[pre,post]$ interval for each vertex
is marked next to each vertex.  The $postlist$ for this run is \textcolor{blue}{$postlist = (v_4,v_7,v_1,v_2,v_3,v_5,v_6)$}.  }
\label{fig:dfs}
\end{figure}



\emph{Proof.} Let $u^* := \arg\min_{u\in C_i\cup C_j} pre[u]$, i.e., $u^*$ is the first explored vertex in $C_i\cup C_j$.
Consider the two cases. 

\begin{figure}[h!]
\centering{\input{meta}}
\caption{Two cases in proving above claim.}
\label{fig:meta}
\end{figure}

First, assume that $u^* \in C_i$. Then $u^*$ can reach all vertices in $C_i\cup C_j\setminus \{u^*\}$.
Hence, all vertices in $C_i\cup C_j\setminus \{u^*\}$ will be explored \emph{within} exploring $u^*$.
In other words, for any vertex $v\in C_i\cup C_j\setminus \{u^*\}$, the interval $[pre[v], post[v]]$
is a subset of $[pre[u^*], post[u^*]]$. This results in two facts:
$\max_{u\in C_i} post[u] = post[u^*]$ and $\max_{v\in C_j} post[v] < post[u^*]$.
Combined, we have that $\max_{u\in C_i} post[u] > \max_{v\in C_j} post[v]$.

Second, assume that $u^* \in C_j$. Then $u^*$ can \emph{not} reach any vertex in $C_i$; otherwise
$C_i\cup C_j$ form a single connected component, conflicting to the fact that any connected component must be maximal.
Hence, all vertices in $C_i$ will remain unexplored after exploring $u^*$.
In other words, for any vertex $v\in C_i$, the interval $[pre[v], post[v]]$
locates after~(and disjoint with) $[pre[u^*], post[u^*]]$. This gives that
$\max_{u\in C_i} post[u] > post[u^*]$. Besides, 
we also have $\max_{v\in C_j} post[v] = post[u^*]$ as all vertices in $C_j\setminus \{u^*\}$ will be examined within exploring $u^*$.
Combined, we have that $\max_{u\in C_i} post[u] > \max_{v\in C_j} post[v]$.
\qed

The above key claim directly suggest a property about the structure of the meta-graph, given below.

\begin{corollary}
The list of all connected components in $V_M$ sorted in decreasing order of $\max_{u\in C_i} post[u]$, $1\le i \le k$,
is a linearization of $G_M$.
\label{col1}
\end{corollary}

Verify this is the case in Figure~\ref{fig:dfs}. Answer: such list is $(C_4, C_3, C_1, C_2)$, and it is a linearization of $G_M$.

Above property also suggests us to consider an \emph{ordering of vertices} in descending post values.
Specifically, we define \emph{postlist} as such ordering of vertices.
Notice that postlist can be stored in an array and can be efficiently constructed in DFS-with-time.
Please refer to Lecture A10 for its pseudo-code.
See Figure~\ref{fig:dfs} for an example.

As a direct consequence of Corollary~\ref{col1} and the definition of postlist, we have the following property.
This is true as ``the first appearance'' exactly gives the ``largest post value'' since the postlist is sorted 
in descending order of post values.

\begin{corollary}
The list of all connected components in $V_M$
sorted by their first appearance of in the postlist is a linearization of $G_M$.
\label{col2}
\end{corollary}

Verify this is the case in Figure~\ref{fig:dfs}. Answer: following $postlist$ the corresponding list of 
connected components is $(C_4, C_4, C_3, C_1, C_3, C_2, C_3)$.
The first appearance of each component is $(C_4, C_3, C_1, C_2)$, which is a linearization of $G_M$
and exactly the one that is sorted by their largest post value.

%%
%%How about the first vertex of $postlist$? Does it in a sink vertex of $G_M$?
%%No. And Figure~\ref{fig:dfs} gives such an example: $v_6$ is the first vertex of $postlist$
%%and it is not in a sink vertex of $G_M$.
%%Note though, if $G$ is a DAG, then the first vertex of $postlist$ must be a sink of $G$,
%%as we prove that in DAGs, reverse of $postlist$ is a linearization of $G$ and the last vertex of a linearization must be a sink.

To summarize, now we can build a postlist that satisfies above Corollary~\ref{col2}.
This almost achieves the desired property of the special ordering:
``the ordering of connected components sorted by their first appearance in the
special ordering should form a reverse-linearization of the meta-graph.''
The only difference is that postlist implies a \emph{linearization} of the meta-graph
while special ordering requires a \emph{reverse-linearization} of the meta-graph.
Below, we introduce \emph{reverse graph} to fill this gap.

\section*{Reverse Graph}

\begin{definition}
Let $G = (V,E)$ be a directed graph. The \emph{reverse graph} of $G$, denoted as $G_R = (V, E_R)$,
has the same set of vertices and edges with reversed direction, i.e., $(u,v) \in E$ if and only if $(v,u)\in E_R$.
\end{definition}

\begin{figure}[h!]
\centering{\input{reverse}}
\caption{Graph and its reverse graph, and the corresponding meta-graphs.}
\label{fig:reverse}
\end{figure}

Following properties can be easily proved using above definition.

\begin{property}
There is a path from $u$ to $v$ in $G$ if and only if there is a path from $v$ to $u$ in $G_R$.
In other words, $u$ can reach $v$ in $G$ if and only if $u$ can be reached from $v$ in $G_R$.
\end{property}

\begin{property}
$(G_R)_R = G$.
\end{property}

\begin{property}
A vertex $v$ of DAG $G$ is a source vertex if and only if $v$ is a sink vertex of $G_R$.
\end{property}


\begin{property}
$G$ and $G_R$ has the same set of connected components. %In other words the meta-graph of $G$ has the same set of vertices with the meta-graph of $G_R$.
\end{property}


\begin{property}
The meta-graph of $G_R$ is the reverse graph of the meta-graph of $G$. Formally, $(G_R)^M = (G^M)_R$.
\end{property}

\begin{property}
$X$ is a linearization of DAG $G$ if and only if the reverse of $X$ is a linearization of $G_R$~(or, $X$ is a reverse-linearization of $G_R$).
In particular, since any meta-graph is a DAG, we have that 
$X$ is linearization of $G^M$ if and only if $X$ is a reverse-linearization of $(G^M)_R = (G_R)^M$,
and that $X$ is a linearization of $(G_R)^M = (G^M)_R$ if and only if $X$ is a reverse-linearization of $G^M$.
\end{property}


\section*{Algorithm to Determine Connected Components in Directed Graphs}

Combining above DFS-with-timing and reverse graph, here is the pseudo-code we use to obtain a special ordering.
The key is that we run DFS-with-timing on the reverse graph $G_R$, instead of on the given graph $G$.

\begin{minipage}{0.8\textwidth}
	\aaA {4}{Algorithm to determine the \textcolor{blue}{specific order}}\xxx
	\aab {build $G_R$ of $G$;}\xxx
	\aab {run DFS with timing on $G_R$ to get $postlist$;}\xxx
	\aab {return $postlist$;}\xxx
	\aaa {end algorithm;}\xxx
\end{minipage}

We now show that, above $postlist$, i.e., the $postlist$ obtained by running DFS-with-timing on $G_R$,
satisfies the condition that ``the ordering of connected components sorted by their first appearance in above
$postlist$ forms a reverse-linearization of the meta-graph $G_M$''.
In fact, since above $postlist$ is obtained by running DFS-with-timing on $G_R$,
according to Corollary~\ref{col2}, 
the list of all connected components sorted by their first appearance of in above $postlist$ is a linearization of $(G_R)^M$.
According to Property~6 of reverse graph, we know that a linearization of $(G_R)^M$ is a reverse-linearization of $G_M$.

Above $postlist$ can then be used by the DFS algorithm as a special order to obtain all connected components, with pseudo-code copied below.

\begin{minipage}{0.8\textwidth}
	\aaA {8}{function DFS ($G = (V, E)$)}\xxx
	\aab {num-cc = 0;}\xxx
	\aaB {5}{for \textcolor{blue}{$v_i$ in the $postlist$ obtained above}}\xxx
	\aaC {3}{if ($visited[i] = 0$)}\xxx
	\aad {num-cc = num-cc + 1;}\xxx
	\aad {explore ($G, v_j$);}\xxx
	\aac {end if;}\xxx
	\aab {end for;}\xxx
	\aaa {end algorithm;}\xxx
\end{minipage}

\begin{minipage}{0.8\textwidth}
	\aaA {5}{function explore ($G = (V, E), v_i \in V$)}\xxx
	\aab {$visited[i] = $ num-cc;}\xxx
	\aaB {2}{for any edge $(v_i, v_j) \in E$}\xxx
	\aac {if ($visited[j] = 0$): explore ($G, v_j$);}\xxx
	\aab {end for;}\xxx
	\aaa {end algorithm;}\xxx
\end{minipage}


The entire algorithm runs in $\Theta(|V| + |E|)$ time.
